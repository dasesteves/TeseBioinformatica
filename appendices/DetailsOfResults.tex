\chapter{Details of Results}
\label{app:details_results}

This appendix consolidates evidence and materials referenced throughout the dissertation that would otherwise disrupt the flow if included inline. It also provides a practical roadmap for inserting new evidence as it becomes available, aligned with the improvement analysis (docs/ and tmp\_ai\_reports/).

\section{Roadmap for Evidence Insertion}
\begingroup\sloppy
The following items specify where to place each type of evidence when ready. Each item cites the target chapter/section and recommended figure/table label.
\begin{itemize}
    \item As-is architecture diagram (SCMVV): Introduction (Section~\ref{sec:as_is_architecture}); Figure label \texttt{\seqsplit{fig:as\_is\_architecture\_scmvv}}. Include: core legacy systems (AIDA-PCE, ADSE, SONHO, SCLINICO, CEGID/PRIMAVERA, ... where applicable, national interfaces such as \gls{pem}); data stores (Oracle schemas, other DBs/fileshares); integration mechanisms (APIs, manual/CSV exchanges); identity/auth context (LDAP/SSO if present); known failure points and manual handoffs. Caption must state consolidated source (docs/tmp\_ai\_reports) and anonymization note.
    \item Current medication process swimlane: Introduction (Section~\ref{sec:current_process_org}); Figure label \texttt{\seqsplit{fig:as\_is\_swimlane\_scmvv}}. Lanes: Physician (prescription), Pharmacy (validation/stock), Nursing (administration/recording), Systems/Records (AIDA-PCE, other systems, paper). Mark handoffs, feedback loops, transcription points, and timing anchors (order entry, validation, dispensing, administration). Caption must state consolidated source (docs/tmp\_ai\_reports) and anonymization note.
    \item Baseline artefacts (legacy screens, anonymized extracts): Results (Baseline sections), with table/figure labels \texttt{\seqsplit{fig:baseline\_*}} or \texttt{\seqsplit{tab:baseline\_*}}.
    \item Qualitative before/after comparison table: Results (Table~\ref{tab:before_after_qualitative}); update cells with evidence.
    \item Performance snapshots (API latencies, reliability): Results (Indicative Performance), figures \texttt{\seqsplit{fig:perf\_*}}.
    \item User acceptance artifacts (SUS summary, interview quotes): Results (Stakeholder Feedback), tables/figures \texttt{\seqsplit{tab:sus\_*}}, \texttt{\seqsplit{fig:feedback\_*}}.
    \item ROI/Cost elements (if applicable): Expected Results (Financial Impact), Figure~\ref{fig:roi-analysis} notes.
\end{itemize}
\endgroup

\section{Evidence Checklist (docs/ and tmp\_ai\_reports/)}
To ensure coverage without overclaiming, collect the following (no counts here):
\begin{itemize}
    \item Representative legacy process artifacts: prescription, validation, administration records; stock movement entries (fields only). For fields, include article identifiers, movement type, timestamps, lot/expiry, and user role where applicable.
    \item Screenshots/mockups: legacy AIDA-PCE views relevant to the cycle; new unified screens for the same steps.
    \item Workflow timing anchors: available timestamps at order entry, validation, dispensing, administration; note systems where each anchor is available.
    \item Handoff indicators: examples of pharmacist-prescriber clarifications; typical turnaround points.
    \item Data coherence samples across systems: patient IDs, active medication lists, stock positions, highlighting discrepancies.
    \item Usability notes linked to specific screens: navigation friction, duplicate entry, missing alerts.
\end{itemize}

\section{Baseline Artefacts: Image Naming and Placement Map}
To streamline insertion, follow this naming/mapping for anonymized baseline artefacts (legacy extracts/screens):
\begingroup\emergencystretch=1.5em
\begin{itemize}
    \item Stock movements extract (legacy fields): \texttt{\seqsplit{images/generated/baseline\_prf\_movements\_extract\_v1.png}} → Results Table/\ref{tab:baseline_prf_movements_fields} caption reference.
    \item Prescription fields snapshot (legacy): \texttt{\seqsplit{images/generated/baseline\_prescription\_fields\_v1.png}} → Results Table/\ref{tab:baseline_prescription_fields} caption reference.
    \item Validation queue snapshot (legacy): \texttt{\seqsplit{images/generated/baseline\_validation\_queue\_v1.png}} → Results Baseline section (new figure label: \texttt{fig:baseline\_validation\_queue}).
    \item Administration record snapshot (legacy/eMAR): \texttt{\seqsplit{images/generated/baseline\_administration\_record\_v1.png}} → Results Baseline section (new figure label: \texttt{fig:baseline\_administration\_record}).
    \item Prescription leaflet before validation (legacy): \texttt{\seqsplit{images/generated/baseline\_prescription\_leaflet\_v1.png}} → Results Table/\ref{tab:baseline_prescription_leaflet_fields} caption reference.
    \item Pre-validation patient list (legacy): \texttt{\seqsplit{images/generated/baseline\_patient\_list\_pre\_validation\_v1.png}} → Results Table/\ref{tab:baseline_patient_list_pre_validation_fields} caption reference.
    \item Pre-prescription patient list (legacy): \texttt{\seqsplit{images/generated/baseline\_patient\_list\_pre\_prescription\_v1.png}} → Results Table/\ref{tab:baseline_patient_list_pre_prescription_fields} caption reference.
    \item Frequency schedule (legacy): \texttt{\seqsplit{images/generated/baseline\_presc\_freq\_v1.png}} → Results Table/\ref{tab:baseline_prescription_frequency_fields}.
    \item Route dictionary (legacy): \texttt{\seqsplit{images/generated/baseline\_route\_dictionary\_v1.png}} → Results Table/\ref{tab:baseline_route_dictionary_fields}.
\end{itemize}
\endgroup

Captions must cite consolidated source (docs/tmp\_ai\_reports) and state “anonymized”.

\section{Screenshot/Mockup Capture Specification}
This section details what to capture for side-by-side comparisons (legacy vs. new) and how to name/place files. Use mockups only when real screenshots are not available, always with anonymization and disclosure in captions.
\subsection*{General Requirements}
\begingroup\emergencystretch=1.5em
\begin{itemize}
    \item Anonymize all PHI: patient identifiers, dates, staff names, bed numbers. Blur/redact before exporting.
    \item Filename convention: \texttt{\seqsplit{images/generated/step\_NAME\_(legacy|new)\_v1.png}} (e.g., \texttt{\seqsplit{step\_prescription\_legacy\_v1.png}}).
    \item Insert in Chapter~\ref{chap:Results}, Section ``System Demonstration'', under ``Per-step evidence placeholders''.
    \item Captions: include source (docs/tmp\_ai\_reports or institutional docs), and note \textquotedblleft anonymized\textquotedblright{} or \textquotedblleft mockup\textquotedblright{}.
\end{itemize}
\endgroup

\subsection*{Steps and Screens to Capture}
\begin{itemize}
    \item Prescription (AIDA-PCE vs. unified UI): fields visible (drug, dose, route, frequency), interaction checks (if any), user context.
    \item Pharmaceutical validation (legacy handoff/records vs. validation queue in new system): queue/list, decision actions, audit markers.
    \item Administration (paper/eMAR variability vs. standardized eMAR flow): task list, administration confirmation, barcode step (if applicable).
    \item Stock movements (AIDA-PCE/PRF vs. new PRF module): movement entry form (article code, lot/expiry, qty, timestamp, user role), stock view.
    \item Cross-system touches (optional): SONHO billing linkage, CEGID/PRIMAVERA inventory decrement, SClínico clinical record update (illustrate the need for multi-system recording).
\end{itemize}

\subsection*{Minimum Shots per Step}
\begin{itemize}
    \item Legacy: one representative screen per step plus a detail zoom (if needed for fields).
    \item New: one equivalent screen per step plus a detail zoom of improvements (checks, audit trail, single sign-on context).
\end{itemize}

\subsection*{Placement Map (labels)}
\begin{itemize}
    \item Prescription: Figures \texttt{fig:prescription\_legacy}, \texttt{fig:prescription\_new}.
    \item Validation: Figures \texttt{fig:validation\_legacy}, \texttt{fig:validation\_new}.
    \item Administration: Figures \texttt{fig:administration\_legacy}, \texttt{fig:administration\_new}.
    \item Stock: Figures \texttt{fig:stock\_legacy}, \texttt{fig:stock\_new}.
    \item Security/Compliance evidence: Tables \texttt{tab:audit\_logs\_completo\_fields}, \texttt{tab:audit\_logs\_cc\_man\_fields}, \texttt{tab:users\_directory\_fields}.
    \item Workflow timing anchors: Tables \texttt{tab:timing\_prescription\_lifecycle}, \texttt{tab:timing\_episode\_context}.
\end{itemize}

\subsection*{Mockup Guidance}
When screenshots are not possible, mirror only fields/flows already described in docs/ and tmp\_ai\_reports. Avoid inventing new features. Use neutral sample text and include a “mockup” note in captions.

\section{Notes on Anonymization and Compliance}
\begingroup\sloppy
All artifacts inserted must be anonymized and comply with GDPR and institutional policies. Where real screenshots cannot be shown, use faithful mockups and describe the original fields/flows.
\endgroup

\section{Outstanding Improvements (from docs/ and tmp\_ai\_reports/)}
This section centralizes planned improvements identified in the analysis. Each item notes the target chapter/section and what will be inserted (no content invented here; placeholders remain until evidence is available).
\begin{itemize}
    \item ROI model details (Contribution, Financial Impact): specify inputs/assumptions (development/maintenance costs, avoided ADE costs, time savings, licensing deltas), method notes, and sensitivity analysis plan. Link to Figure~\ref{fig:roi-analysis}.
    \item eMAR/BCMA alignment (State of the Art, Section~\ref{sec:emar_bcma}; Methodology/Contribution): add bedside administration flow alignment notes; produce mockups if actual screens unavailable.
    \item National context references (State of the Art, Section~\ref{sec:national_context_portugal}): add 1--2 references to SNS/SPMS guidance or reports, once confirmed.
    \item Related implementations (State of the Art): add short paragraph citing case studies of unified front-ends over legacy systems, if identified.
    \item KPI operational definitions (Contribution, Section~\ref{sec:KPIs}): define error categories, cycle-time measurement points, adoption and reliability indicators (definitions only).
    \item Security/compliance evidence (Results, Security and Compliance Posture): add configuration snippets and process notes (authN/authZ, encryption, audit), and ethics approval metadata (identifier/date) when available.
    \item Screenshots/mockups per step (Results): legacy vs. new for prescription, validation, administration; ensure captions include source and anonymization notes.
    \item Baseline anonymized extracts (Results): controlled-substance movement fields, sample record structures; no counts included.
\end{itemize}

\section{Cross-reference and Labels Validation Checklist}
Before final compilation, confirm the following labels and references resolve correctly and point to the intended figures/tables/sections.
\begin{itemize}
    \item Introduction: \texttt{fig:as\_is\_architecture\_scmvv}, \texttt{fig:as\_is\_swimlane\_scmvv}; Sections~\ref{sec:context_scmvv}, \ref{sec:as_is_architecture}, \ref{sec:current_process_org}.
    \item State of the Art: Sections~\ref{sec:emar_bcma}, \ref{sec:process_standardization}, \ref{sec:national_context_portugal}.
    \item Contribution (Expected Results): Figure~\ref{fig:architecture}, Figure~\ref{fig:roi-analysis}, Figure~\ref{fig:future-roadmap}; Section~\ref{sec:KPIs}.
    \item Results: Table~\ref{tab:before_after_qualitative}; baseline sections and appendix reference~\ref{app:details_results}.
    \item Discussion: references to as-is baseline (Sections~\ref{sec:as_is_architecture}, \ref{sec:current_process_org}).
\end{itemize}

\section{Mock Artifact Production Guidance}
When real evidence is not available, produce mockups with the following constraints:
\begin{itemize}
    \item Faithfully reflect fields and flows described in docs/ and tmp\_ai\_reports/; avoid introducing features not mentioned.
    \item Include caption notes with source (“consolidated from docs/tmp\_ai\_reports”) and a disclosure that the image is a mockup.
    \item Store images under \texttt{images/generated/} and replace placeholders when real artifacts become available.
\end{itemize}