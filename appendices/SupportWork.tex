\chapter{Support Work}
\label{app:support_work}

This appendix consolidates auxiliary materials that support the main text but would otherwise interrupt its flow. It includes scripts, data dictionaries, interface notes and training aids referenced in the analysis (docs/ and tmp\_ai\_reports/).

\section{Data Extraction and Analysis Aids}
\begin{itemize}
    \item Legacy data field inventories for prescriptions, validations, administrations, and stock movements (structure only; no counts). Source: institutional docs and tmp\_ai\_reports.
    \item Example request/response shapes for legacy APIs or export views when applicable (anonymized).
    \item Notes on data reconciliation steps used to construct baseline snapshots.
\end{itemize}

\section{Interface and Integration Notes}
\begin{itemize}
    \item Authentication/authorization configuration checklists (LDAP/\gls{sso}, \gls{jwt}).
    \item Integration touchpoints considered (billing, reporting, national platforms) with interface placeholders to be filled when available.
    \item Error handling and audit logging conventions adopted in the artefact.
\end{itemize}

\section{Training and Change Management Aids}
\begin{itemize}
    \item Role-specific quick reference guides (to be populated): prescriber, pharmacist, nurse.
    \item Sprint feedback form templates and issue triage workflows.
    \item Communication plan outline and champion responsibilities.
\end{itemize}

\section{Usability and Evaluation Instruments (Templates)}
\subsection{SUS Questionnaire Summary Template}
\begin{table}[H]
    \centering
    \caption{Template: SUS responses summary (to be filled post-evaluation).}
    \label{tab:template_sus}
    {\setlength{\tabcolsep}{4pt}\small
    \begin{tabularx}{\textwidth}{@{}p{3cm} p{3cm} >{\raggedright\arraybackslash}X@{}}
        \toprule
        \textbf{Participant ID} & \textbf{Role} & \textbf{SUS item responses (1--5) and total} \\
        \midrule
        P001 & Nurse & items 1--10; total score; notes \\
        P002 & Pharmacist & items 1--10; total score; notes \\
        P003 & Physician & items 1--10; total score; notes \\
        \bottomrule
    \end{tabularx}}
\end{table}

\subsection{Interview/Focus Group Guide Outline}
\begin{itemize}
    \item Perceived usability and clarity of workflows (by role).
    \item Decision support usefulness and alert fatigue (if any).
    \item Handoffs and communication improvements.
    \item Data entry burden and duplication changes.
    \item Suggestions and barriers to adoption (training needs, policies).
\end{itemize}

\subsection{Observation Checklist (Point-of-Care)}
\begin{itemize}
    \item Steps executed from prescription to administration (timestamps where visible).
    \item System transitions (legacy/new) and manual transcriptions.
    \item Interruptions and rework instances; error prevention prompts.
    \item Any deviations from standard operating procedures.
\end{itemize}

\section{Data Dictionary Templates}
The following templates standardize the capture of field definitions and mappings (structure only; no identifiable data).

\subsection{Entity Fields Template}
\begin{table}[H]
    \centering
    \caption{Template: Entity fields (to be filled when source information is available).}
    \label{tab:template_entity_fields}
    {\setlength{\tabcolsep}{4pt}\small
    \begin{tabularx}{\textwidth}{@{}P{2.7cm} P{2.3cm} P{2.7cm} P{2.2cm} P{1.6cm} Y@{}}
        \toprule
        \textbf{Field} & \textbf{Source System} & \textbf{Entity/Table} & \textbf{Data Type} & \textbf{Nullable} & \textbf{Description / Notes} \\
        \midrule
        name & AIDA-PCE & \path{prf_movimentos} & VARCHAR2(\ldots) & No & Movement descriptor (example placeholder) \\
        code & AIDA-PCE & \path{prf_artigos} & NUMBER(\ldots) & No & Article identifier (example placeholder) \\
        timestamp & AIDA-PCE & \path{prf_movimentos} & DATE & No & Event time (example placeholder) \\
        user\_role & AIDA-PCE & \path{prf_movimentos} & VARCHAR2(\ldots) & Yes & Actor role at action time (example placeholder) \\
        \bottomrule
    \end{tabularx}}
\end{table}

\subsection{Field Mapping Template}
\begin{table}[H]
    \centering
    \caption{Template: Source-to-target field mapping (to be filled when integration is defined).}
    \label{tab:template_field_mapping}
    {\setlength{\tabcolsep}{2.5pt}\small
    \begin{tabularx}{\textwidth}{@{}P{2.2cm} P{2.7cm} P{2.7cm} P{2.2cm} P{2.1cm} Y@{}}
        \toprule
        \textbf{Source System} & \textbf{Source Entity.\-Field} & \textbf{Target Entity.\-Field} & \textbf{Transform/\-Rule} & \textbf{Validation} & \textbf{Notes} \\
        \midrule
        AIDA-PCE & \path{prf_movimentos.codigo} & \path{stock_movements.article_code} & Normalize code (upper) & Must exist in catalog & Placeholder \\
        AIDA-PCE & \path{prf_movimentos.data} & \path{stock_movements.event_ts} & TZ-aware convert & Not null & Placeholder \\
        AIDA-PCE & \path{prf_movimentos.utilizador} & \path{audit.actor} & Map to role/user id & Exists in users & Placeholder \\
        \bottomrule
    \end{tabularx}}
\end{table}

All entries must be derived from institutional documentation and/or tmp\_ai\_reports summaries, with full anonymization and without inserting record-level data.

\section{Reference Backlog (to be populated)}
This backlog lists references to be added/confirmed (e.g., national guidance, related implementations). Each entry should capture citation key, short note, and target chapter/section.
\begin{itemize}
    \item SNS/SPMS guidance on interoperability/security: target State of the Art (Section~\ref{sec:national_context_portugal}).
    \item Case studies of unified front-ends over legacy HIS: target State of the Art (Related Implementations).
    \item eMAR/BCMA implementation reports in similar contexts: target State of the Art (Section~\ref{sec:emar_bcma}).
    \item ROI unit cost sources for ADE and time-savings: target Contribution (Financial Impact).
\end{itemize}