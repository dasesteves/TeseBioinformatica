\chapter{Applications}
\label{chap:Applications}

This chapter details the methodological framework that guided this research. It begins by outlining the high-level research paradigm and strategy, then elaborates on the specific design of the study, the development methodology employed, and the methods used for data collection and evaluation. The chapter concludes with a discussion of ethical considerations and the inherent limitations of the study.

\section{Research Paradigm and Strategy}

This research adopts a \textit{pragmatic paradigm}, integrating quantitative and qualitative methods to address the complex, real-world challenges of hospital medication management \cite{venkatesh2003}. The work is fundamentally grounded in \textit{Design Science Research (\gls{dsr})}, an approach that emphasizes the creation and evaluation of an innovative artifact—in this case, an integrated software system—to solve a concrete organizational problem \cite{martin2017}. This paradigm is ideal as it provides a rigorous structure for developing a technologically sound solution while ensuring its practical relevance and utility within the specific context of the \gls{scmvv} hospital.

To operationalize the DSR paradigm, an \textit{Action Research} strategy was employed \cite{greenhalgh2017}. This choice was dictated by the dynamic nature of the clinical environment, which required an iterative and adaptive approach. Action Research involves continuous cycles of planning, acting, observing, and reflecting, allowing for the incremental improvement of the system based on empirical feedback gathered directly from healthcare professionals. By making practitioners active partners in the research, this strategy fosters a co-creation of knowledge and ensures the final artifact is deeply aligned with user needs and clinical workflows.

\section{Research Design and Execution}

The project was structured to answer a set of core research questions concerning the impact and implementation of integrated clinical systems. The primary questions guiding this study were: 1) How can an integrated system effectively reduce medication errors? 2) What are the critical success factors for its adoption? 3) How can its multifaceted impact be rigorously evaluated?

To answer these, the project was executed in a series of structured phases, as outlined in the work plan (Chapter~\ref{chap:WorkPlan}). The initial \textit{Analysis and Planning} phase (Jan-Feb 2025) was dedicated to requirement elicitation and a deep analysis of the legacy AIDA-PCE system. This involved conducting semi-structured interviews with 15 key stakeholders (physicians, nurses, pharmacists), performing 40 hours of direct workflow observation, and analyzing a dataset of 10,000 historical prescriptions. The outputs were a formal Software Requirements Specification (SRS) and detailed process maps, which informed the system's high-level architecture.

\subsection{Development and Implementation Methodology}

The system was developed using an adapted \textit{agile methodology}, blending principles from user-centered design and rapid prototyping to facilitate continuous engagement with clinicians \cite{fowler2018}. The development work was divided into focused implementation modules.

The \textit{Core Infrastructure Development} (Mar-Apr 2025) involved setting up development environments and implementing the data access layer and a secure, \gls{jwt}-based authentication system. This was followed by the development of the primary clinical modules: the \textit{User Management and Treatment Registration Module} (May-Jun 2025) and the \textit{Pharmacy and Prescription Validation Module} (Jul-Aug 2025), which included the integration of a real-time clinical decision support engine (\gls{cdss}).

A critical component of the methodology was the integration with external and legacy systems during the \textit{External System Integrations} phase (Sep-Oct 2025). This required careful mapping of data schemas and ensuring real-time data synchronization with platforms such as \gls{sonho} (for billing), \gls{adse} (for insurance), and the national e-prescription platform (\gls{pem}).

Finally, the \textit{Optimization, Testing, and Validation} phase (Nov-Dec 2025) involved comprehensive load testing to ensure the system could support over 500 concurrent users, performance profiling to guarantee API response times under 200ms, and formal User Acceptance Testing (UAT) to confirm readiness for clinical use.

\subsection{Risk Management Strategy}

A proactive risk management strategy was integral to the methodology. As detailed in the Risk Analysis section of the Work Plan (Section~\ref{sec:RiskAnalysis}), key identified risks included resistance to change from staff, technical incompatibilities with legacy systems, and potential system performance degradation. Mitigation strategies were implemented for each. For instance, to counter resistance to change, a comprehensive change management plan was executed, featuring continuous training and the appointment of departmental "champions" to advocate for the new system. To de-risk technical challenges, extensive integration testing was conducted in a dedicated staging environment that mirrored production, and the system was designed with built-in fault tolerance, including offline modes for critical functionalities.

\section{Data Collection and Evaluation}

To evaluate the system's impact, a mixed-methods approach to data collection was used, gathering both quantitative and qualitative data during the six-month pilot study.

\subsection{Quantitative Data Collection}
Quantitative data focused on objective, measurable indicators of performance and safety. System performance metrics, such as response time and uptime, were continuously monitored. Clinical process data, including medication error rates and task completion times, were collected and compared against baseline data from the legacy system. Usage metrics, including active user counts and feature adoption rates, were also tracked to gauge user engagement.

\subsection{Qualitative Data Collection}
Qualitative data provided rich, contextual insights into the user experience. In-depth, semi-structured interviews were conducted with healthcare professionals and hospital managers to understand their perceptions of the system's impact on their work. Furthermore, direct participant observation of clinical workflows before and after implementation allowed for an assessment of how the system was integrated into practice and what unintended consequences or workarounds emerged.

\subsection{Evaluation Criteria}
The system's success was assessed against a predefined set of criteria rooted in the Donabedian model for quality of care, focusing on structure, process, and outcomes. The specific Key Performance Indicators (KPIs) derived from these criteria are detailed in Chapter~\ref{chap:ExpectedResults} (Section~\ref{sec:KPIs}).

For \textit{Patient Safety}, the primary criterion was a statistically significant reduction in medication errors. For \textit{Operational Efficiency}, success was defined by measurable reductions in process cycle times and improved interdisciplinary communication. For \textit{User Acceptance}, the evaluation relied on achieving a "Good" or "Excellent" score on the System Usability Scale (SUS) and overwhelmingly positive qualitative feedback, along with high adoption rates across all clinical groups.

\section{Ethical Considerations and Limitations}

\subsection{Ethical Protocol}
The study protocol received full approval from the Ethics Committee of the SCMVV. All research activities adhered strictly to the General Data Protection Regulation (GDPR) \cite{european2016}. Patient data was fully anonymized before analysis, and informed consent was obtained from all participating healthcare professionals. Robust technical and procedural safeguards were implemented to protect data confidentiality and integrity.

\subsection{Limitations of the Study}
The findings must be interpreted in light of several methodological and technical limitations. The single-center design at SCMVV may limit the generalizability of the results to other hospital contexts. The six-month evaluation period, while sufficient for initial assessment, does not capture long-term effects on organizational culture or patient outcomes. The pre-post comparison, lacking a parallel control group, cannot definitively exclude the influence of confounding variables. Finally, the system's reliance on a central Oracle database and its partial, rather than full, conformance with the HL7 FHIR standard represent technical constraints that offer clear directions for future work.