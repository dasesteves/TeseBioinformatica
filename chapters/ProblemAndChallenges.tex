\chapter{Discussion}
\label{chap:ProblemAndChallenges}

This chapter provides a critical analysis of outcomes and implications, contextualizing their significance within the scientific literature and the operational realities of the Portuguese National Health Service (SNS). It examines implications of the findings (and preliminary evidence), foreseeable challenges of implementation, and the inherent limitations of the study's design, concluding with broader implications for clinical practice, hospital management, and future research in healthcare informatics.

\section{Interpretation of Expected Implications}

The central thesis of this work is that a strategically designed, unified frontend architecture serves as a catalyst for overcoming systemic fragmentation in hospital information systems. Emerging evidence from controlled trials points to reduction in medication errors and improvements in workflow efficiency. However, interpretation must transcend raw metrics. Literature-aligned targets (e.g., large relative reductions) are treated as benchmarks to be validated rather than as final pilot outcomes, reinforcing \textit{user-centered design principles} in mitigating clinical risk \cite{ciapponi2021,radley2013}. Observations are discussed against the as-is baseline (Sections~\ref{sec:as_is_architecture} and~\ref{sec:current_process_org}).

Similarly, improvements in system performance and user satisfaction provide evidence that modernizing the user-facing layer of technology can yield high returns even when legacy backends remain partially in place. A strategic lesson follows: high-impact modernization does not always require a complete, high-risk "rip-and-replace" overhaul of the entire infrastructure \cite{adler2021}. The observed behavior of a microservices-based architecture reinforces the value of architectural flexibility and incremental deployment in complex, risk-averse environments \cite{newman2021}.

\section{Anticipated Challenges and Contextualization}

Successful implementation hinges on navigating significant sociotechnical challenges, particularly within the high-pressure context of the Portuguese public healthcare system \cite{goiana2024portuguese}. While technical hurdles in integrating with legacy systems are considerable \cite{keasberry2017}, primary challenges are human and organizational. Introducing a new system to clinical staff already facing workload pressures requires a change management strategy that is empathetic, inclusive, and demonstrates immediate value \cite{rogers2003}.

Project success depends on effective application of user-centered co-design, ensuring clinicians are active partners in design and rollout \cite{venkatesh2003}. Resistance rooted in established workflows and cognitive fatigue is expected. The mitigation strategy relies on an agile, iterative implementation that enables rapid feedback and adjustment, empowering clinical champions and demonstrating tangible workflow improvements from early stages \cite{may2013}. This approach directly confronts systemic fragmentation, where lack of integration forces clinicians to become "human middleware" bridging information gaps \cite{pinto2016identification}.

\section{Limitations and Avenues for Future Research}

The findings of this study must be interpreted within the boundaries of its methodological design, which present clear avenues for future research. The single-center design, while necessary for a deep, context-specific implementation at SCMVV, inherently limits the statistical generalizability of the findings to other institutions with different organizational cultures or technical infrastructures. The quasi-experimental design, lacking a parallel control group, means that while we can measure significant improvements, we cannot definitively exclude the influence of confounding variables.

Furthermore, the evaluation focuses on objective metrics of patient safety and operational efficiency. It is acknowledged that new information systems affect psychosocial dimensions of work, including cognitive load and potential for burnout among healthcare professionals \cite{hertzum2022}. A detailed analysis of these factors, while important, is outside the defined scope of this dissertation and remains a direction for future investigation.

Technically, while the proposed architecture promotes interoperability, this initial phase does not achieve full conformance with standards such as HL7 FHIR. Achieving semantic interoperability is a next step, paving the way for seamless data exchange with national health platforms and other providers \cite{mandl2020}.

Despite these limitations, this work is poised to make significant contributions. For clinical practice, it will offer a validated model for modernizing critical hospital workflows. For management, it will present a data-driven case for investing in user-experience-focused technology. For research, it will lay the groundwork for future studies on long-term impacts, scalability, and the broader effects of technological change on the healthcare workforce. 