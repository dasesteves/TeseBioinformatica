\chapter*{Resumo}
\addcontentsline{toc}{chapter}{Resumo}

A fragmentação dos sistemas de informação no Serviço Nacional de Saúde português representa um desafio sistémico à segurança do doente e à eficiência operacional, particularmente no ciclo do medicamento. Este projeto de dissertação propõe-se a endereçar este problema no contexto da Santa Casa da Misericórdia de Vila Verde (SCMVV) através do desenho, desenvolvimento e avaliação de uma plataforma de software centralizada. O objetivo é unificar os fluxos de trabalho clínico-farmacêuticos, atualmente dispersos por múltiplos sistemas legados, numa única interface de utilizador moderna e coesa.

Adotando uma metodologia de \textit{Design Science Research} (DSR), o projeto irá criar um artefacto tecnológico — um sistema web com uma arquitetura de microsserviços (Node.js) e um frontend reativo (TypeScript/React) — concebido para se integrar com a infraestrutura existente. A avaliação do sistema será focada em indicadores de desempenho chave (KPIs) específicos, antecipando-se uma redução significativa dos erros de medicação e um aumento da eficiência dos processos para enfermeiros e farmacêuticos. A contribuição principal deste trabalho será a validação de um modelo de modernização sociotécnica que, se bem-sucedido, poderá servir de referência para outras unidades de saúde que enfrentam desafios de fragmentação semelhantes.

\vspace{6mm}
\noindent\textbf{Palavras-chave:} Sistemas de Informação em Saúde, Segurança do Doente, Gestão da Medicação, Design Science Research, Unificação de Sistemas, Interoperabilidade Clínica.

\vspace*{\fill}

\chapter*{Abstract}
\addcontentsline{toc}{chapter}{Abstract}

The fragmentation of information systems within the Portuguese National Health Service constitutes a systemic challenge to patient safety and operational efficiency, particularly in the medication management lifecycle. This dissertation project aims to address this problem in the context of the Santa Casa da Misericórdia de Vila Verde (SCMVV) by designing, developing, and evaluating a centralized software platform. The primary objective is to unify the clinical-pharmaceutical workflows, currently fragmented across multiple legacy systems, into a single, modern, and cohesive user interface.

Adopting a \textit{Design Science Research} (DSR) methodology, the project will create a technological artifact—a web-based system featuring a microservices architecture (Node.js) and a reactive frontend (TypeScript/React)—designed to integrate with the existing infrastructure. The system's evaluation will focus on specific Key Performance Indicators (KPIs), with the anticipation of achieving a significant reduction in medication errors and an increase in process efficiency for nurses and pharmacists. The main contribution of this work will be the validation of a sociotechnical modernization model that, if successful, could serve as a reference for other healthcare institutions facing similar fragmentation challenges.

\vspace{6mm}
\noindent\textbf{Keywords:} Health Information Systems, Patient Safety, Medication Management, Design Science Research, System Unification, Clinical Interoperability. 